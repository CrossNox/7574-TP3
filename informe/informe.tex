\documentclass[titlepage,a4paper,oneside]{article}
\usepackage[utf8]{inputenc}
\usepackage{amsmath}
\usepackage{mathabx}
\usepackage{graphicx}
\usepackage{minted}
\usepackage{booktabs}
\usepackage[english,spanish,es-noindentfirst,es-nosectiondot,es-nolists,
es-noshorthands,es-lcroman,es-tabla]{babel}
\usepackage{lmodern}             % Use Latin Modern fonts
\usepackage[T1]{fontenc}         % Better output when a diacritic/accent is used
\usepackage[utf8]{inputenc}      % Allows to input accented characters
\usepackage{textcomp}            % Avoid conflicts with siunitx and microtype
\usepackage{microtype}           % Improves justification and typography
\usepackage[svgnames]{xcolor}    % Svgnames option loads navy (blue) colour
\usepackage[hidelinks,urlcolor=blue]{hyperref}
\hypersetup{colorlinks=true, allcolors=Navy, pdfstartview={XYZ null null 1}}
\newtheorem{lemma}{Lema}
\usepackage[width=14cm,left=3.5cm,marginparwidth=3cm,marginparsep=0.35cm,
height=21cm,top=3.7cm,headsep=1cm, headheight=1.6cm,footskip=1.2cm]{geometry}
\usepackage{csquotes}
\usepackage{biblatex}
\addbibresource{informe.bib}
\usepackage[pdf]{graphviz}


\begin{document}

\begin{titlepage}
\title{
	75.74 \-- Distribuidos I \-- TP3\\
    \large Facultad de Ingeniería\\
	Universidad de Buenos Aires
}
\author{
	Mermet, Ignacio Javier\\
	\texttt{98153}
}
\date{Junio 2022}

\maketitle

\end{titlepage}

\tableofcontents

\newpage

\section{Sobre la entrega}
El código de la entrega se puede encontrar en \href{https://github.com/CrossNox/7574-TP3}{GitHub}.
- Mencionar tecnologías y que vean el readme para ve como se ejecuta.

\section{Analsis del trabajo a realizar}

\subsection{Puntos clave}
% TODO: los cuatro que nos dijo Eze

\section{Hipótesis, simplificaciones y asunciones}
- El server de rabbit no va a estar replicado

\section{Arquitectura}
- Lógica
- De Procesos
- De desarrollo
- Fisica

\subsection{Componentes}

\subsection{Servidor de cara al cliente}
- Explicar protocolo

\subsubsection{Workers Stateless}

\subsubsection{Workers Stateful}
\textbf{Remoción de duplicados}:


\subsubsection{Revividor}

\subsubsection{Storage API}

\section{Algoritmo de elección de líder}

\section{Middleware}

\section{Procesamiento de mensajes}
- Mencionar que es at-least-once
- Mostrar diagrama de secuencia

\section{Cliente}
- Explicar el protocolo
- Que conoce las addresses de preferencia de los servers de cara al cliente

\printbibliography

\end{document}
